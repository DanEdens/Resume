% Copyright 2013 Christophe-Marie Duquesne <chmd@chmd.fr>
% Copyright 2014 Mark Szepieniec <http://github.com/mszep>
% 
% ConText style for making a resume with pandoc. Inspired by moderncv.
% 
% This CSS document is delivered to you under the CC BY-SA 3.0 License.
% https://creativecommons.org/licenses/by-sa/3.0/deed.en_US

\startmode[*mkii]
  \enableregime[utf-8]  
  \setupcolors[state=start]
\stopmode

\setupcolor[hex]
\definecolor[titlegrey][h=757575]
\definecolor[sectioncolor][h=397249]
\definecolor[rulecolor][h=9cb770]

% Enable hyperlinks
\setupinteraction[state=start, color=sectioncolor]

\setuppapersize [A4][A4]
\setuplayout    [width=middle, height=middle,
                 backspace=20mm, cutspace=0mm,
                 topspace=10mm, bottomspace=20mm,
                 header=0mm, footer=0mm]

%\setuppagenumbering[location={footer,center}]

\setupbodyfont[11pt, helvetica]

\setupwhitespace[medium]

\setupblackrules[width=31mm, color=rulecolor]

\setuphead[chapter]      [style=\tfd]
\setuphead[section]      [style=\tfd\bf, color=titlegrey, align=middle]
\setuphead[subsection]   [style=\tfb\bf, color=sectioncolor, align=right,
                          before={\leavevmode\blackrule\hspace}]
\setuphead[subsubsection][style=\bf]

\setuphead[chapter, section, subsection, subsubsection][number=no]

%\setupdescriptions[width=10mm]

\definedescription
  [description]
  [headstyle=bold, style=normal,
   location=hanging, width=18mm, distance=14mm, margin=0cm]

\setupitemize[autointro, packed]    % prevent orphan list intro
\setupitemize[indentnext=no]

\defineitemgroup[enumerate]
\setupenumerate[each][fit][itemalign=left,distance=.5em,style={\feature[+][default:tnum]}]

\setupfloat[figure][default={here,nonumber}]
\setupfloat[table][default={here,nonumber}]

\setuptables[textwidth=max, HL=none]
\setupxtable[frame=off,option={stretch,width}]

\setupthinrules[width=15em] % width of horizontal rules

\setupdelimitedtext
  [blockquote]
  [before={\setupalign[middle]},
   indentnext=no,
  ]


\starttext

\section[title={Johnny Coder},reference={johnny-coder}]

\thinrule

\startblockquote
In this style, the resume starts with a blockquote, where you can
briefly list your specialties, or include a salient quote. Ending a line
with a backslash forces a line break.
\stopblockquote

\thinrule

\subsection[title={Education},reference={education}]

\startdescription{2010-2014 (expected)}
  {\bf PhD, Computer Science}; Awesome University (MyTown)

  {\em Thesis title: Deep Learning Approaches to the Self-Awesomeness
  Estimation Problem}
\stopdescription

\startdescription{2007-2010}
  {\bf BSc, Computer Science and Electrical Engineering}; University of
  HomeTown (HomeTown)

  {\em Minor: Awesomeology}
\stopdescription

\subsection[title={Experience},reference={experience}]

{\bf Your Most Recent Work Experience:}

Short text containing the type of work done, results obtained, lessons
learned and other remarks. Can also include lists and links:

\startitemize
\item
  First item
\item
  Item with \useURL[url1][http://www.example.com][][link]\from[url1].
  Links will work both in the html and pdf versions.
\stopitemize

{\bf That Other Job You Had}

Also with a short description.

\subsection[title={Technical
Experience},reference={technical-experience}]

\startdescription{My Cool Side Project}
  For items which don't have a clear time ordering, a definition list
  can be used to have named items.

  \startitemize[packed]
  \item
    These items can also contain lists, but you need to mind the
    indentation levels in the markdown source.
  \item
    Second item.
  \stopitemize
\stopdescription

\startdescription{Open Source}
  List open source contributions here, perhaps placing emphasis on the
  project names, for example the {\bf Linux Kernel}, where you
  implemented multithreading over a long weekend, or {\bf node.js} (with
  \useURL[url2][http://nodejs.org][][link]\from[url2]) which was
  actually totally your idea\ldots{}
\stopdescription

\startdescription{Programming Languages}
  {\bf first-lang:} Here, we have an itemization, where we only want to
  add descriptions to the first few items, but still want to mention
  some others together at the end. A format that works well here is a
  description list where the first few items have their first word
  emphasized, and the last item contains the final few emphasized terms.
  Notice the reasonably nice page break in the pdf version, which
  wouldn't happen if we generated the pdf via html.

  {\bf second-lang:} Description of your experience with second-lang,
  perhaps again including a {[}link{]}
  \useURL[url3][https://github.com/githubuser/superlongprojectname][][ref]\from[url3],
  this time placing the url reference elsewhere in the document to
  reduce clutter (see source file).

  {\bf obscure-but-impressive-lang:} We both know this one's pushing it.

  Basic knowledge of {\bf C}, {\bf x86 assembly}, {\bf forth},
  {\bf Common Lisp}
\stopdescription

\subsection[title={Extra Section, Call it Whatever You
Want},reference={extra-section-call-it-whatever-you-want}]

\startitemize
\item
  Human Languages:

  \startitemize[packed]
  \item
    English (native speaker)
  \item
    ???
  \item
    This is what a nested list looks like.
  \stopitemize
\item
  Random tidbit
\item
  Other sort of impressive-sounding thing you did
\stopitemize

\thinrule

\startblockquote
\useURL[url4][mailto:email@example.com][][email@example.com]\from[url4]
• +00 (0)00 000 0000 • XX years old\crlf
address - Mytown, Mycountry
\stopblockquote

\stoptext
